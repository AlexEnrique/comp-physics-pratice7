\input{std.tex}

\begin{document}
\title{Prática 7 - Questão 1}
\author{Alex Enrique Crispim}
\maketitle

O método de Euler é o método mais simples para resolução de equações diferenciais ordinarias (EDO's). Uma extensão do método se dá pelo que é chamado \textit{método de Euler-Cromer}, cuja diferença está, basicamente, na ordem de atualização entre a derivada primeira (a qual chamaremos velocidade) e a função em busca.

Para a EDO
\begin{equation*}
  \dv{x}{t} = -x^3 + \sin(t),
\end{equation*}
temos apenas o método de Euler, dos citados anteriormente.

Seja $v_i = \dv{x}{t} (t_i)$ e $x_i = x(t_i)$, podemos utilizar o método das diferenças finitas e escrever
\begin{flalign*}
  x_{i+1} &= x_i + \tau v_i , \\ 
  &= x_i + \tau (sin(t_i) - x_i^3).
\end{flalign*}




\end{document}
