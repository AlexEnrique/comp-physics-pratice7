\documentclass[aps,twocolumn,secnumarabic,balancelastpage,amsmath,amssymb,nofootinbib,floatfix]{revtex4-1}

\usepackage[colorlinks=true,linkcolor=blue]{hyperref}

\usepackage{mathexam}
\usepackage{booktabs}

%\usepackage{a4wide}
\usepackage[utf8]{inputenc}
\usepackage{amsmath}
\usepackage{amsfonts}
\usepackage{amssymb}
\usepackage{mathtools}
\usepackage[brazil]{babel}
%quebra de linha do sumário
%\usepackage[breaklinks=true]{hyperref}
%\usepackage{braket}
\usepackage{minitoc}
\usepackage{wrapfig}
\usepackage{subfigure}
\usepackage{setspace}
\usepackage{underscore}
\usepackage{indentfirst}
\usepackage{physics}

\usepackage{accents}

\usepackage{blindtext}

\usepackage{graphicx}



%\onehalfspacing


\newcommand*{\dt}[1]{%
  \accentset{\mbox{\large\bfseries .}}{#1}}
\newcommand*{\ddt}[1]{%
  \accentset{\mbox{\large\bfseries .\hspace{-0.25ex}.}}{#1}}
\newcommand*{\dddt}[1]{%
  \accentset{\mbox{\large\bfseries .\hspace{-0.25ex}.\hspace{-.20ex}}}{#1}}
\newcommand*{\dnt}[2][4]{\dt{#2}^{\tiny(#1)}}
\newcommand{\esimo}{-\text{ésimo}}
\newcommand{\esima}{-\text{ésima}}

\makeatletter
\newcommand*{\gnuplotinput}[2][1.0]{%
  \begingroup
  \let\@gnplt@input@includegraphics=\includegraphics
  \def\includegraphics##1{\@gnplt@input@includegraphics[scale=#1]{#2}}%
  \let\@gnplt@input@picture=\picture
  \def\picture{\unitlength=#1\unitlength\relax\@gnplt@input@picture}%
  \input{#2}%
  \endgroup
}
\makeatother


\makeatletter
\newcommand*{\customclear}{%
  \close@column@grid
  \cleardoublepage
  \twocolumngrid
}
\makeatother


\usepackage{listings}
\usepackage{color}

\definecolor{mygreen}{rgb}{0,0.6,0}
\definecolor{mygray}{rgb}{0.85,0.85,0.85}
\definecolor{mymauve}{rgb}{1,0.502,0}
\definecolor{numGray}{rgb}{0.3,0.3,0.3}


\lstdefinestyle{CStyle}{language=C,
  backgroundcolor=\color{mygray},   % choose the background color; you must add \usepackage{color} or \usepackage{xcolor}; should come as last argument
  basicstyle=\footnotesize,        % the size of the fonts that are used for the code
  breakatwhitespace=false,         % sets if automatic breaks should only happen at whitespace
  breaklines=true,                 % sets automatic line breaking
  captionpos=b,                    % sets the caption-position to bottom
  commentstyle=\color{mygreen},    % comment style
  deletekeywords={...},            % if you want to delete keywords from the given language
  escapeinside={\%*}{*)},          % if you want to add LaTeX within your code
  extendedchars=true,              % lets you use non-ASCII characters; for 8-bits encodings only, does not work with UTF-8
  frame=single,	                   % adds a frame around the code
  keepspaces=true,                 % keeps spaces in text, useful for keeping indentation of code (possibly needs columns=flexible)
  keywordstyle=\color{blue},       % keyword style
  language=C,                 % the language of the code
  morekeywords={*,...},            % if you want to add more keywords to the set
  numbers=left,                    % where to put the line-numbers; possible values are (none, left, right)
  numbersep=5pt,                   % how far the line-numbers are from the code
  numberstyle=\tiny\color{numGray}, % the style that is used for the line-numbers
  rulecolor=\color{black},         % if not set, the frame-color may be changed on line-breaks within not-black text (e.g. comments (green here))
  showspaces=false,                % show spaces everywhere adding particular underscores; it overrides 'showstringspaces'
  showstringspaces=false,          % underline spaces within strings only
  showtabs=false,                  % show tabs within strings adding particular underscores
  stepnumber=1,                    % the step between two line-numbers. If it's 1, each line will be numbered
  stringstyle=\color{mymauve},     % string literal style
  tabsize=2,	                   % sets default tabsize to 2 spaces
  title=\lstname                   % show the filename of files included with \lstinputlisting; also try caption instead of title
}


\lstdefinestyle{PyStyle}{language=Python,
  backgroundcolor=\color{mygray},   % choose the background color; you must add \usepackage{color} or \usepackage{xcolor}; should come as last argument
  basicstyle=\footnotesize,        % the size of the fonts that are used for the code
  breakatwhitespace=false,         % sets if automatic breaks should only happen at whitespace
  breaklines=true,                 % sets automatic line breaking
  captionpos=b,                    % sets the caption-position to bottom
  commentstyle=\color{mygreen},    % comment style
  deletekeywords={...},            % if you want to delete keywords from the given language
  escapeinside={\%*}{*)},          % if you want to add LaTeX within your code
  extendedchars=true,              % lets you use non-ASCII characters; for 8-bits encodings only, does not work with UTF-8
  frame=single,	                   % adds a frame around the code
  keepspaces=true,                 % keeps spaces in text, useful for keeping indentation of code (possibly needs columns=flexible)
  keywordstyle=\color{blue},       % keyword style
  language=Python,                 % the language of the code
  morekeywords={*,...},            % if you want to add more keywords to the set
  numbers=left,                    % where to put the line-numbers; possible values are (none, left, right)
  numbersep=5pt,                   % how far the line-numbers are from the code
  numberstyle=\tiny\color{numGray}, % the style that is used for the line-numbers
  rulecolor=\color{black},         % if not set, the frame-color may be changed on line-breaks within not-black text (e.g. comments (green here))
  showspaces=false,                % show spaces everywhere adding particular underscores; it overrides 'showstringspaces'
  showstringspaces=false,          % underline spaces within strings only
  showtabs=false,                  % show tabs within strings adding particular underscores
  stepnumber=1,                    % the step between two line-numbers. If it's 1, each line will be numbered
  stringstyle=\color{mymauve},     % string literal style
  tabsize=2,	                   % sets default tabsize to 2 spaces
  title=\lstname                   % show the filename of files included with \lstinputlisting; also try caption instead of title
}


\begin{document}
  \begin{center}
    \LARGE \textbf{Física Computacional} \\
    \Large \textbf{Tarefa 7 - Questão 2} \\
    \large Alex Enrique Crispim
  \end{center}

  Buscamos, agora, resolver a mesma EDO
  \begin{equation*}
    x^\prime (t) = f(t, x) , \\
    x(0) = x_0,
  \end{equation*}
  por meio do \textit{Método de Runge-Kutta}. O método consiste em tomar o ponto médio do intervalo para o qual fizemos diferenças finitas, quando usamos Euler, para determinar um ponto intermediário de tal forma que possamos reduzir a ordem do erro.

  Quando utilizamos o método de Euler, de modo implicito, utilizamos uma aproximação via Polinômio de Taylo (série de Taylor truncada) de primeira ordem. De outra forma, tomamos a série
  \begin{equation*}
    x(t + h) = x(t) + h x^\prime (t) + \frac{h^2}{2!} x^{\prime \prime}(t) + ...,
  \end{equation*}
  até o termo linear em $h$ resultando em $x(t+h) = x(t) + h x^\prime (t)$. O método de Runge-Kutta busca seguir a mesma ideia de aproximação por Taylor, porém truncando a série no termo de ordem quadrátrica em $h$, de tal forma que o erro passa a ser $\order{h^3}$.

  O termo de primeira ordem reproduz o método de Euler; é simplesmente $h f$ (omitiremos os argumentos daqui para frente e as derivadas parciais para $f$ serão denotadas com subscritos como $f_t, \ f_{txx}$). O termo de segunda ordem pode ser obtido da seguinte forma:
  \begin{equation*}
    x^{\prime \prime} (t) = \dv{f}{t} = \pdv{f}{t} \dv{t}{t} + \pdv{f}{x} \dv{x}{t} = f_t + f_x f,
  \end{equation*}
  produzindo
  \begin{flalign*}
    x(t+h) &= x + h f + \frac{h^2}{2} \qty(f_t + f_x f) + \order{h^3}, \\
           &= x + h f + \frac{1}{2} h^2 f_t + \frac{1}{2} h^2 f f_x + \order{h^3}.
  \end{flalign*}

  Podemos agora expandir $\dv*{f}{t}$ em Taylor para que os termos da forma $f_t + f f_x$ sejam escritos em termos de $f$, usando a aproximação de Euler ($x(t+h) = x + hf$).
  \begin{equation*}
    f(t + \alpha h, x + \beta h f) = f + \alpha h f_t + \beta h f f_x + \frac{1}{2} \qty(\alpha h \pdv{t} + \beta h \pdv{x})^2 f(\bar{t}, \bar{x}).
  \end{equation*}

  Utilizamos dois parâmetros ($\alpha$ e $\beta$) a serem ajustados de forma a produzir a aproximação desejada por Taylor (melhor aproximação). O erro da aproximação em primeira ordem é guardado no termo final (quadrático em $h$).

  Se escrevemos nossa expansão para $x(t+h)$ como
  \begin{equation*}
    x(t+h) = x + w_1 h f + w_2 h f(t + \alpha h, x + \beta h f),
  \end{equation*}
  podemos utilizar a expansão anterior no termo à direita de $w_2$ e comparar com a expansão em Taylor original para ajustar os parâmetros inserido.

  \begin{equation}
    x(t + h) = x + (w_1 + w_2) h f + \alpha w_2 h^2 f_t + \beta w_2 f f_x + \order{h^3}.
    \label{eq:1}
  \end{equation}

  Por comparação da expressão de Taylor obtida com (\ref{eq:1}), temos as seguintes relações:
  \begin{flalign*}
    & w_1 + w_2 = 1 \qc \alpha w_2 = \frac{1}{2} \qc \beta w_2 = \frac{1}{2}, \\
    & \alpha = 1 \qc \beta = 1 \qc w_1 = w_2 = \frac{1}{2}.
  \end{flalign*}

  Assim, podemos escrever
  \begin{equation}
    x(t + h) = x(t) + \frac{1}{2} (K_1 + K_2),
    \label{eq:2}
  \end{equation}
  \begin{equation*}
    \begin{cases}
      K_1 = h f(t,x), \\
      K_2 = h f(t + h, x + K_1).
    \end{cases}
  \end{equation*}

  Abaixo apresentamos o algoritmo, de forma mais direta, para o método de Runge-Kutta.

  \begin{figure}[h]
    \center
    \begin{tk3}
    \begin{tikzpicture}[node distance = 15 mm, auto]
      
    \end{tikzpicture}
    \end{tk3}
  \end{figure}


\end{document}
