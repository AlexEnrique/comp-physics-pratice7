\input{../std.tex}
\newenvironment{tk3} {
\tikzstyle{loop} = [regular polygon, regular polygon sides=6, shape aspect=0.3, minimum width=1cm, minimum height=1cm, draw,scale=.7, align=center, text width=0.9	cm, fill = blue!15]

\tikzstyle{startstop} = [rectangle, rounded corners, minimum width=3cm, minimum height=.7cm,text centered, draw=black, fill=red!30, text width = 6cm]

\tikzstyle{process} = [rectangle, minimum width=3cm, minimum height=1cm, text centered, draw=black, fill=orange!30, text width = 4cm]

\tikzstyle{processSmall} = [rectangle, minimum width=1cm, minimum height=1cm, text centered, draw=black, fill=orange!30, text width = 2cm]

\tikzstyle{decision} = [diamond, draw, fill=yellow!30,
    text width=4.5em, text badly centered, node distance=3cm, inner sep=0pt]

\tikzstyle{line} = [draw, -latex']

\tikzstyle{cloud} = [draw, ellipse,fill=red!20, node distance=3cm, minimum height=2em, text width = 4cm]

\tikzstyle{io} = [trapezium, trapezium left angle=70, trapezium right angle=-70, text centered, text width = 3.5cm, minimum height=1cm, minimum width=2cm, draw=black, fill=blue!30]

\tikzstyle{arrow} = [thick,->,>=stealth]
\tikzstyle{line} = [draw, -latex']
} {  }


\begin{document}
  \begin{center}
    \LARGE \textbf{Física Computacional} \\
    \Large \textbf{Tarefa 7 - Questão 2} \\
    \large Alex Enrique Crispim
  \end{center}

  Buscamos, agora, resolver a mesma EDO
  \begin{equation*}
    x^\prime (t) = f(t, x) , \\
    x(0) = x_0,
  \end{equation*}
  por meio do \textit{Método de Runge-Kutta}. O método consiste em tomar o ponto médio do intervalo para o qual fizemos diferenças finitas, quando usamos Euler, para determinar um ponto intermediário de tal forma que possamos reduzir a ordem do erro.

  Quando utilizamos o método de Euler, de modo implicito, utilizamos uma aproximação via Polinômio de Taylo (série de Taylor truncada) de primeira ordem. De outra forma, tomamos a série
  \begin{equation*}
    x(t + h) = x(t) + h x^\prime (t) + \frac{h^2}{2!} x^{\prime \prime}(t) + ...,
  \end{equation*}
  até o termo linear em $h$ resultando em $x(t+h) = x(t) + h x^\prime (t)$. O método de Runge-Kutta busca seguir a mesma ideia de aproximação por Taylor, porém truncando a série no termo de ordem quadrátrica em $h$, de tal forma que o erro passa a ser $\order{h^3}$.

  O termo de primeira ordem reproduz o método de Euler; é simplesmente $h f$ (omitiremos os argumentos daqui para frente e as derivadas parciais para $f$ serão denotadas com subscritos como $f_t, \ f_{txx}$). O termo de segunda ordem pode ser obtido da seguinte forma:
  \begin{equation*}
    x^{\prime \prime} (t) = \dv{f}{t} = \pdv{f}{t} \dv{t}{t} + \pdv{f}{x} \dv{x}{t} = f_t + f_x f,
  \end{equation*}
  produzindo
  \begin{flalign*}
    x(t+h) &= x + h f + \frac{h^2}{2} \qty(f_t + f_x f) + \order{h^3}, \\
           &= x + h f + \frac{1}{2} h^2 f_t + \frac{1}{2} h^2 f f_x + \order{h^3}.
  \end{flalign*}

  Podemos agora expandir $\dv*{f}{t}$ em Taylor para que os termos da forma $f_t + f f_x$ sejam escritos em termos de $f$, usando a aproximação de Euler ($x(t+h) = x + hf$).
  \begin{equation*}
    f(t + \alpha h, x + \beta h f) = f + \alpha h f_t + \beta h f f_x + \frac{1}{2} \qty(\alpha h \pdv{t} + \beta h \pdv{x})^2 f(\bar{t}, \bar{x}).
  \end{equation*}

  Utilizamos dois parâmetros ($\alpha$ e $\beta$) a serem ajustados de forma a produzir a aproximação desejada por Taylor (melhor aproximação). O erro da aproximação em primeira ordem é guardado no termo final (quadrático em $h$).

  Se escrevemos nossa expansão para $x(t+h)$ como
  \begin{equation*}
    x(t+h) = x + w_1 h f + w_2 h f(t + \alpha h, x + \beta h f),
  \end{equation*}
  podemos utilizar a expansão anterior no termo à direita de $w_2$ e comparar com a expansão em Taylor original para ajustar os parâmetros inserido.

  \begin{equation}
    x(t + h) = x + (w_1 + w_2) h f + \alpha w_2 h^2 f_t + \beta w_2 f f_x + \order{h^3}.
    \label{eq:1}
  \end{equation}

  Por comparação da expressão de Taylor obtida com (\ref{eq:1}), temos as seguintes relações:
  \begin{flalign*}
    & w_1 + w_2 = 1 \qc \alpha w_2 = \frac{1}{2} \qc \beta w_2 = \frac{1}{2}, \\
    & \alpha = 1 \qc \beta = 1 \qc w_1 = w_2 = \frac{1}{2}.
  \end{flalign*}

  Assim, podemos escrever
  \begin{equation}
    x(t + h) = x(t) + \frac{1}{2} (K_1 + K_2),
    \label{eq:2}
  \end{equation}
  \begin{equation}
    \begin{cases}
      K_1 = h f(t,x), \\
      K_2 = h f(t + h, x + K_1).
    \end{cases}
    \label{eq:3}
  \end{equation}

  Abaixo apresentamos o algoritmo, de forma mais direta, para o método de Runge-Kutta.

  \begin{figure}[h]
    \center
    \begin{tk3}
    \begin{tikzpicture}[node distance = 15 mm, auto]
      \node (start) [startstop] {Algoritmo de Runge-Kutta};
      \node (initializing) [process, below of = start] {Inicializa $f(t, x)$, $t  \leftarrow t_0$, $t_f$, $x_0$ e $h$};
      \node (CalcKs) [process, below of = initializing] {Calcula $K_1$ e $K_2$ via (\ref{eq:3}).};
      \node (newX) [process, below of = CalcKs] {$x_{i+1} = x_i + \frac{1}{2} (K_1 + K_2)$; \\ $t \leftarrow t + h$};
      \node (dec) [decision, right of = newX, xshift = 8mm] {$(t > t_f)$?};
      \node (printX) [io, below of = newX] {Imprime \texttt{x[]} em função de $t$.};
      \node (stop) [startstop, below of = printX] {Finalizado};

      \draw [arrow] (start) -- (initializing);
      \draw [arrow] (initializing) -- (CalcKs);
      \draw [arrow] (CalcKs) -- (newX);
      \draw [arrow] (newX) -- (dec);
      \path [line]  (dec) -- (3.8,-3.0) -- node [anchor = south]{não} (CalcKs);
      \path [line]  (dec) -- (3.8,-6.0) -- node [anchor = north]{sim} (printX);
      \draw [arrow] (printX) -- (stop);
    \end{tikzpicture}
    \end{tk3}
    \caption{Algoritmo de Runge-Kutta para a solução de EDO's}
  \end{figure}

  Uma implementação do algoritmo descrito acima pode ser encontrado na pasta \textit{question 2} do endereço: \url{https://github.com/AlexEnrique/comp-physics-pratice7}.


\end{document}
