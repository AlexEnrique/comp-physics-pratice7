\input{../std.tex}
\newenvironment{tk3} {
\tikzstyle{loop} = [regular polygon, regular polygon sides=6, shape aspect=0.3, minimum width=1cm, minimum height=1cm, draw,scale=.7, align=center, text width=0.9	cm, fill = blue!15]

\tikzstyle{startstop} = [rectangle, rounded corners, minimum width=3cm, minimum height=.7cm,text centered, draw=black, fill=red!30, text width = 6cm]

\tikzstyle{process} = [rectangle, minimum width=3cm, minimum height=1cm, text centered, draw=black, fill=orange!30, text width = 4cm]

\tikzstyle{processSmall} = [rectangle, minimum width=1cm, minimum height=1cm, text centered, draw=black, fill=orange!30, text width = 2cm]

\tikzstyle{decision} = [diamond, draw, fill=yellow!30,
    text width=4.5em, text badly centered, node distance=3cm, inner sep=0pt]

\tikzstyle{line} = [draw, -latex']

\tikzstyle{cloud} = [draw, ellipse,fill=red!20, node distance=3cm, minimum height=2em, text width = 4cm]

\tikzstyle{io} = [trapezium, trapezium left angle=70, trapezium right angle=-70, text centered, text width = 3.5cm, minimum height=1cm, minimum width=2cm, draw=black, fill=blue!30]

\tikzstyle{arrow} = [thick,->,>=stealth]
\tikzstyle{line} = [draw, -latex']
} {  }

\begin{document}
  \begin{center}
    \LARGE \textbf{Física Computacional} \\
    \Large \textbf{Tarefa 7 - Questão 3} \\
    \large Alex Enrique Crispim
  \end{center}

  O método de Runge-Kutta de quarta ordem segue uma ideia semelhante ao caso feito para a questão 2, porém tomando a expansão até quarta ordem.

  Seguindo o mesmo procedimento, com um pouco mais de álgebra, chegamos às seguintes equações:
  \begin{equation}
    x(t+ h) = x(t) + \frac{1}{6} \qty(K_1 + 2K_2 + 2K_3 + K_4).
    \label{eq:1}
  \end{equation}
  \begin{equation}
    \begin{cases}
      K_1 &= h f(t, x) , \\
      K_2 &= h f(t + \frac{1}{2} h, x + \frac{1}{2} K_1), \\
      K_3 &= h f(t + \frac{1}{2} h, x + \frac{1}{2} K_2), \\
      K_4 &= h f(t + h, x + K_3).
    \end{cases}
    \label{eq:2}
  \end{equation}

  O algoritmo para o método Runge-Kutta-4 é praticamente o mesmo para o Runge-Kutta-2, como mostrado abaixo.
  \begin{figure}[h]
    \center
    \begin{tk3}
    \begin{tikzpicture}[node distance = 15 mm, auto]
      \node (start) [startstop] {Algoritmo de Runge-Kutta de quarta ordem};
      \node (initializing) [process, below of = start] {Inicializa $f(t, x)$, $t  \leftarrow t_0$, $t_f$, $x_0$ e $h$};
      \node (CalcKs) [process, below of = initializing] {Calcula $K_1$, $K_2$, $K_3$ e $K_4$ via (\ref{eq:2}).};
      \node (newX) [process, below of = CalcKs] {$x_{i+1} = x_i + \frac{1}{2} (K_1 + 2K_2 + 2K_3 + K_4)$; (eq. \ref{eq:1}) \\ $t \leftarrow t + h$};
      \node (dec) [decision, right of = newX, xshift = 8mm] {$(t > t_f)$?};
      \node (printX) [io, below of = newX] {Imprime \texttt{x[]} em função de $t$.};
      \node (stop) [startstop, below of = printX] {Finalizado};

      \draw [arrow] (start) -- (initializing);
      \draw [arrow] (initializing) -- (CalcKs);
      \draw [arrow] (CalcKs) -- (newX);
      \draw [arrow] (newX) -- (dec);
      \path [line]  (dec) -- (3.8,-3.0) -- node [anchor = south]{não} (CalcKs);
      \path [line]  (dec) -- (3.8,-6.0) -- node [anchor = north]{sim} (printX);
      \draw [arrow] (printX) -- (stop);
    \end{tikzpicture}
    \end{tk3}
    \caption{Algoritmo de RK4 para a solução de EDO's}
  \end{figure}

  Uma implementação do algoritmo RK4 pode ser encontrado na pasta \textit{question 3} do endereço: \url{https://github.com/AlexEnrique/comp-physics-pratice7}.

  É importante mencionar que, como o próprio nome do método sugere, o algoritmo RK4 tem seu erro de quinta ordem em $h$ (a série é truncada em quarta ordem). Isso significa uma convergência muito mais rápida em comparação com RK2 (Runge-Kutta de segunda ordem). Como exemplo, considere $h = 10^{-2}$. Para RK2 temos um erro de ordem $\order{10^{-6}}$, enquanto que para RK4 o erro é $\order{10^{-10}}$.

  O grande problema com o algoritmo RK4 se dá quando a EDO envolvida é tal que as soluções se distanciam entre si conforme a variável $t$ cresce. Por exemplo, para a EDO $x^\prime = 10x + 11 t - 5t^2 - 1 $, com $x(0) = 0$, o método RK4 nos dá um resultado errado. As figuras abaixo mostram as soluções exatas (vermelho) e via RK4 para a EDO acima.
  \begin{figure}[h]
    \center
    \includegraphics[scale = .45]{Figure1}
    \includegraphics[scale = .45]{Figure2}
  \end{figure}

  O problema se deve ao fato de que, uma variação $h$ nas condições iniciais leva a uma variação muito grande dos valores esperados para a função. Considere por exemplo a EDO dada por
  \begin{equation*}
    x^\prime = f(t, x), \\
    x(0) = s.
  \end{equation*}

  Se trocamos $s$ por $s+h$ (o que é induzido, por exemplo, por um erro de arredondamento), devemos considerar a função $x(t)$ como a depender do parâmetro $s$ da forma de uma variável, escrevendo $x(t, s)$ no lugar. Podemos expandir $x(t, s+h)$ em Taylor da seguinte forma
  \begin{equation*}
    x(t, s+h) = x(t,s) + h \pdv{s} x(t, s) + \order{h^2}.
  \end{equation*}

  A divergência entre a função esperada $x(t, s)$ e $x(t, s+h)$ pode ser escrita como
  \begin{equation*}
    \lim_{t \to \infty} \abs{x(t, s+h) - x(t,s)} = \infty \Leftrightarrow \lim_{t \to \infty} \abs{\pdv{s} x(t, s)} = \infty
  \end{equation*}

  Para calcular a derivada parcial de $x$ com relação a $s$, usamos o seguinte procedimento:
  \begin{equation*}
    \pdv{t} \pdv{s} x(t, s) = \pdv{s} \pdv{t} x(t,s) = f_x(t, x(t,s)) \pdv{s} x(t,s) + f_t(t, x(t,s)) \pdv{t}{s},
  \end{equation*}
  como uma variação em $t$ não depende de uma variação em $s$, $\pdv{t}{s} = 0$ e fazendo $u(t) = \pdv{s} x(t,s)$ e $q(t) = f_x(t, x(t,s))$, obtemos a seguinte EDO, da relação anterior:
  \begin{equation*}
    u^\prime = q u ,
  \end{equation*}
  cuja solução é $u = c e^{Q(t)}$, com
  \begin{equation*}
    Q(t) = \int_a^t \dd{\theta} q(\theta).
  \end{equation*}

  A divergência do limite de $\abs{\pdv*{x}{s}}$ se dá então pela divergência de $Q(t)$ (pois leva a diergir ($u$)).

  Se $Q$ for uma função positiva e limitada inferiormente por um valor maior do que zero, é fácil ver que a função diverge:
  \begin{equation*}
    Q(t) = \int_a^t \dd{\theta} q(\theta) > \int_a^t \delta \dd{\theta} = \delta (t-a) \xrightarrow[t \to \infty]{} \infty.
  \end{equation*}

  A conclusão final é: a solução diverge se $f_x > \delta$ e converge quando $f_x < - \delta$, com $\delta > 0$.

  Para a EDO fornecida para as questões 1, 2 e 3, $f(t, x) = - x^3 + \sin(t)$ e, logo, $f_x = -x^2 \leq 0, \forall x$, enquando que a EDO mal-comportada acima leva à $f_x = 10 > 0$, divergindo do valor real.

\end{document}
