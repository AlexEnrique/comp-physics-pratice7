\input{../std.tex}

\begin{document}
  \begin{center}
    \LARGE \textbf{Física Computacional} \\
    \Large \textbf{Tarefa 7 - Questão 6} \\
    \large Alex Enrique Crispim
  \end{center}

  Podemos estender o método de Runge-Kutta descrito nas questões 2 e 3 para mais dimensões fazendo-se a troca dos $K_i$ por arrays $\vb{K}_i$, acompanhando a troca de $x^\prime = f$ por
  \begin{equation*}
    \dv{\vb{x}}{t} = \vb{f}(t, \vb{x}_i).
  \end{equation*}

  Para o método RK4, mas equações têm basicamente a mesma forma, trocando-se os escalares por vetores;
  \begin{equation}
    \vb{x}_{i+1} = \vb{x}_i + \frac{1}{6} \qty[\vb{K}_1 + 2 \vb{K}_2 + 2\vb{K}_3 + \vb{K}_4],
  \end{equation}
  \begin{equation*}
    \begin{cases}
      \vb{K}_1 = h \vb{f}(t_i, \vb{x}_i), \\
      \vb{K}_2 = h \vb{f}(t_i + h/2, \vb{x}_i + \vb{K}_1 / 2), \\
      \vb{K}_3 = h \vb{f}(t_i + h/2, \vb{x}_i + \vb{K}_2 / 2), \\
      \vb{K}_4 = h \vb{f}(t_i + h, \vb{x}_i + \vb{K}_3).
    \end{cases}
  \end{equation*}

  Em certas linguagem (geralmente as que suporta POO), a operação de produto de um array por um escalar e soma de arrays é definida da mesma forma como soma de vetores, de tal forma que podemos escrever as equações acima de forma muito natual. Um exemplo fora feito em \textit{Julia} na pasta \textit{question 6}.

  Como o trabalho não é tão formal, acredito não ter problema de expressar minha opinião a respeito de um primeiro contato com Julia: eu não gostei da linguagem. Algumas poucas coisas que gostei, tem-se implementadas forma melhor ou, no mínimo, quase equivalente em python. Os testes de performance que se encontram no site da linguagem me são duvidosos, após os primeiros usos da mesma. A sintaxe é incomoda em alguns pontos. Por exemplo, a contagem dos índices dos arrays começa em 1. Uma das vantagens de começar a contar a partir do 0 é o fato de pode se pensar como contando de 1, mas utilizando sempre intervalos abertos à direita e fechados à esquerda. A linguagem Julia tirou isso. Além de que a sintaxe de muitas coisas como a função \texttt{append!()} ou \texttt{push!()} para arrays, é, no mínimo, incomoda. 

\end{document}
